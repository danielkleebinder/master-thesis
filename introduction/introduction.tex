\label{sec:introduction}

\newacronym{iot}{IoT}{Internet of Things}

Recent years have shown an ever-growing interest in \gls{bpm} from both industrial and research communities. Members of these can be categorized into sub-groups with very different educational backgrounds, each applying \gls{bpm} in their own way and coming up with their own very specific requirements. Business administrations, for example, aim to use methodical \gls{bpm} to optimize workflows to reduce cost and increase customer satisfaction. Software communities that must implement such workflows using existing technologies strive for robust but flexible solutions to keep up with changing requirements introduced by business administrations or by legal authorities~\cite{weske2012_bpm_introduction}. To bridge the gap between these two worlds, computer science research communities devised notations and formal specifications that create a common ground~\cite{bpmn_v2_spec,weske2012_bpm_process_choreographies,weske2012_bpm_process_orchestration,architecture_for_multi_chain_bp_ladleif}.

Nonetheless, some open problems still need to be solved, especially regarding inter-organizational \gls{bpm} due to a lack of mutual trust between counterparties. It is not uncommon that cooperating participants are in a conflict of interest that hinders the progression toward a set goal. The impact of trust missing as a key ingredient in collaborative business processes like supply chains~\cite{johnston2004_supplier_trust_performance}, health care~\cite{bo_collaboration_between_healthcare_providers_covid_19} or logistics~\cite{fynes2005_impact_of_relationships_on_supply_chain_performance} has been studied thoroughly where all research came to the same conclusion of trust being a factor that cannot be underestimated; neither in social nor inter-organizational relationships~\cite{impact_of_trust_on_supply_chains}. Recent years, however, have shown an innovation emerge that has the potential to revolutionize how trust is handled --- introducing \gls{bct}.

Roughly outlined as a tamper-proof series of timestamped transactions and famously known for cryptocurrencies such as Bitcoin~\cite{nakamoto2009} and Ethereum~\cite{buterin2020}, blockchains are also a viable tool in the arsenal of \gls{bpm}. A lot of potential use cases have been discussed and demonstrated, ranging from the use in workflow execution systems to monitoring of business processes, from governmental environments to the \gls{iot}~\cite{blockchains_for_bpmn_challenges_and_opportunities,untrusted_bp_execution_using_blockchain,lean_architecture_for_blockchain_based_process_execution,interpreted_bp_on_blockchain_loukil,blockchain_and_iot_for_bpm}. However, almost all of these approaches leverage the blockchain as a first-level citizen and directly build on top of it using smart contracts which inevitably leads to privacy concerns and drastically increased cost~\cite{ethereum_yellow_paper}. On blockchains, the execution of workflow steps, that only a subset of participants is concerned with, must nonetheless be executed by all participants in the same way. This exposes potentially confidential information and increases execution cost. Keeping internal processes private, however, is of utmost importance for many companies and organizations~\cite{farah2021_security_of_bps,carminati2018_confidential_bp_execution_on_blockchain}.

Furthermore, concepts tightly coupled to the blockchain, in the form of smart contracts, for example, come with a rather significant flexibility penalty. Changing workflow participants, the workflow definition, or adapting capabilities of the underlying workflow execution system is a complex and costly task~\cite{ethereum_yellow_paper,monitoring_financial_stability_cryptocurrencies}. This means that business processes that require a high level of flexibility can only take advantage of blockchain-intrinsic properties, such as trust decentralization or transparency to some extent which reduces acceptance of these approaches.



\section{Aim of the Work}
\label{sec:introduction:aim_of_the_work}
To tackle the aforementioned open problems in inter-organizational \gls{bpm} and the lack of standardized workflow system architectures~\cite{prybila_master_thesis}, this work proposes a novel concept for a ``time-travelling''\footnote{Being able to jump between active and past states to perform business process validation.} state machine that enables execution of business processes including multiple participants in an environment that provides unconditional mutual trust. The concept aims to provide a modular software architecture that allows the integration of different \glspl{bct} to leverage upon their diverse advantages and to enable integration of future blockchains or layer-2 rollups to prevent participants from being locked to a particular blockchain. Additionally, and in contrast to most existing approaches, the concept provides a high level of flexibility regarding system architecture, workflow participant selection, and workflow definition. To achieve this goal, the following research questions are answered throughout the course of this work.

\begin{enumerate}
    \item[\textbf{RQ1}] What is the state of the art for \gls{bct}-based state machines for business process engines?
    \item[\textbf{RQ2}] Which properties do \gls{bct}-based state machines require to allow time-travel verification of business processes?
    \item[\textbf{RQ3}] Which aspects must be adapted to close the gap between the state of the art and a privacy-preserving \gls{bct}-based state machine that allows time-travel verification?
\end{enumerate}

The first research question is answered in section~\ref{sec:related-work:comparison:rq1} after conducting a thorough literature review in chapter~\ref{sec:related-work}. It should give the reader an overview of existing approaches and identify the gaps in the state of the art. Building upon research question 1, the second research question is answered in section~\ref{sec:ttsm:properties:rq2} by identifying unique properties of the proposed concept derived from the state of the art and requirements identified in real-world industrial use cases. The last research question aims to answer how well the proposed concept can be integrated into existing solutions, a property much expected by users of such a system, and which aspects to adapt in order to make it work.



\section{Methodological Approach}
\label{sec:introduction:methodology}
The following methodologies are employed to answer the research questions, ensure scientific rigor and allow for reproducibility of the results. Tailoring has been applied and described in detail at the beginning of the related work chapter in section~\ref{sec:related-work:methodology} and at the beginning of the chapter that describes the proposed concept in section~\ref{sec:ttsm:methodology}.


\subsection{Literature Review}
\label{sec:introduction:literature_review}
A narrative literature review is performed to accumulate background knowledge of topics including \gls{bpm}, \glspl{bct}, and blockchain-oriented software engineering. Furthermore, work related to workflow execution on the blockchain is investigated and compared with each other in more detail. A predefined set of search words and literature databases such as IEEE and ResearchGate are used to improve reproducibility of the results~\cite{literature_review_rhoades,literature_review_stratton}. The narrative literature review is used to answer RQ1 and lays the foundation for RQ2 and RQ3.


\subsection{Design Science}
\label{sec:introduction:design_science}
The remainder of this work follows the design science research approach for information systems~\cite{hevner2004_design_science}. A concept for a time-travelling state machine is derived in an iterative process. The concept is implemented and evaluated against the state of the art by (1) determining common metrics from related literature in a qualitative analysis process, (2) performing a static analysis of the software architecture to derive its complexity, and (3) simulating real-world use cases in experiments to determine the utility of the produced artifacts. The artifacts produced are (1) a concept for a flexible and robust time-travelling state machine and (2) an instantiation of this concept in the form of a prototype that future work can extend upon and integrate into existing solutions. The research results are communicated to management- and technology-oriented audiences by publishing this work. Design science and the produced artifacts are used to answer RQ2 and RQ3.


\subsection{Software Engineering}
\label{sec:introduction:software_engineering}
Software engineering methodologies are employed during the design, implementation, and evaluation phases. Requirements engineering is used in earlier stages of this work to create and determine industrial real-world use cases that the concept is designed and later evaluated against. BPMN and UML are used throughout the course of the entire work to visualize business processes and software architectures. Other methodologies such as event sourcing, unit- and integration testing, object-oriented, functional and reactive programming are used to implement the prototype. Following standard and state-of-the-art software engineering approaches allow industry and research developers to properly implement a time-travelling state machine, leveraging their existing knowledge base. These methodologies help to answer RQ2 and RQ3.



\section{Structure of the Work}
\label{sec:introduction:structure_of_the_work}
The remainder of this work is structured in a way that introduces the reader to the topics of \gls{bpm} and \glspl{bct}, to then transition into a concept for a time-travelling state machine followed by its evaluation. Background knowledge in chapter~\ref{sec:background} gives a brief introduction into \glspl{bct} and consensus algorithms in section~\ref{sec:background:consensus} followed by a description of software engineering approaches that are applicable for the domain of \glspl{bct}. The remainder of this chapter focuses on notations commonly used in \gls{bpm} in both industrial and research settings.

Thereafter, the reader is introduced to approaches from literature related to this work's contribution in chapter~\ref{sec:related-work}. At the beginning of this chapter, the narrative literature methodology and its tailoring are introduced, followed by detailed descriptions of some of the more relevant approaches found. Each approach is briefly introduced and explained, followed by a short discussion of advantages and disadvantages. The chapter concludes by comparing the presented approaches with the proposed concept of this work in section~\ref{sec:related-work:comparison} and answers RQ1 in subsection~\ref{sec:related-work:comparison:rq1}.

Chapter~\ref{sec:ttsm} then introduces the tailoring applied to the design science methodology in section~\ref{sec:ttsm:methodology}, followed by a thorough description of a time-travelling state machine in section~\ref{sec:ttsm:proposal}. This is the first artifact produced in this work. The second artifact, the instantiation, is described in the prototype design in section~\ref{sec:ttsm:prototype}. It introduces all used technologies, followed by a per-module description of the implementation. This chapter concludes by listing intrinsic properties of a time-travelling state machine as observed in the proposed concept and the prototype design in section~\ref{sec:ttsm:properties} and afterwards answers RQ2 in subsection~\ref{sec:ttsm:properties:rq2}.

The evaluation in chapter~\ref{sec:evaluation} begins with a qualitative analysis where common metrics are derived from related literature and are applied to the proposed concept. This is followed by a static analysis in section~\ref{sec:evaluation:static_analysis} that focuses on formal metrics and the analysis of the software architecture. Afterwards, real-world scenarios are simulated to demonstrate utility in section~\ref{sec:evaluation:simulations}, and the integration into existing systems in section~\ref{sec:evaluation:integration}. Each section summarizes and discusses its results at the end. Chapter~\ref{sec:evaluation} concludes by answering RQ3.

The last chapter~\ref{sec:conclusion} concludes this work by briefly outlining the proposed concept and summarizing the evaluation results. Furthermore, it states problems that remain unsolved, follow-up research questions that came up during the course of this work, and opportunities for future work are also discussed.
